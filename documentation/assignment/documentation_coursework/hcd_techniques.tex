\documentclass[a4wide, 11pt]{article}
\usepackage{a4, fullpage, hyperref, graphicx}
\setlength{\parskip}{0.3cm}
\setlength{\parindent}{0cm}
\graphicspath{ {./images/} }

% This is the preamble section where you can include extra packages etc.

\title{\vspace{-2.0cm}Human Centred Design Techniques Report}

\author{\vspace{-2.0cm}Group 14}

\date{\vspace{-2.0cm}\today}         % inserts today's date

\begin{document}

\maketitle            % generates the title from the data above

% In a few short paragraphs you should discuss and reflect on the Human Centred Design techniques
% that you employed as part of the project. In particular you should discuss your initial research
% into your problem statement (identification of the current state, key design insights and your
% proposed future state) and the HCD Research Methodology employed during the project iterations.
% Additionally, you should diagrammatically evidence the impact of user feedback on your project
% direction during each of the 3 implementation weeks.

\section{HCD Techniques}

\subsection{Initial Research into Problem Statement}

We decided to address the problem of a lack of quality programming education for Key Stage 3 students. After initial research we found that many schools are struggling to keep up with new the UK computing curriculum, where students are now taught Computing from 11-14 as a compulsory course, as there is a lack of skilled programming teachers.

Many existing platforms either do not provide facilities for teachers to see student solutions or provide feedback, or don't give students any direction when attempting to teach challenging concepts.

We envisaged an engaging web application to help students learn difficult progamming concepts that are not taught by other platforms, without discouraging them with unintuitive syntax. This platform would allow teachers to set exercises for their classes to complete and track their students' progress through the app.

\subsection{HCD Research Methodology}

We started the project by creating: personas for our target audience (figure X), a stakeholder map showing which groups of people our app will have an effect on (figure Y) and a mood-o-gram for a teacher's typical programming lesson without our app (figure Z).

Each week we employed an HCD methodology, following roughly the below cycle:....

We employed an HCD methodology each week roughly follows the below cycle:
\begin{itemize}
  \item Initially, we collate the previous week's feedback. We use it to determine what existing features have been most useful to the users and what new features would give them the most benefit in the coming week.
  \item After determining the key features, we develop a mocked up representation of the features. Initially, this would be on paper. However, after creating a vertical slice of the application, we were able to provide mock pages that would externally appear functioning.
  \item Our primary user tester is a teacher - Lucy Glanfield. We have been arranging weekly interviews on the Tuesday where we present our mocked up features and let her play with them, generally with minimal guidance. By observing her unguided interactions we gain valuable information about the usability and intuitiveness of the feature. We then have more open discussions about the feature, application and problem area as a whole. This gives us more direction in terms of what features are needed next and helps us to gather more problem context. When we have ideas about new features, this is when we propose them and gauge how useful they will be.
  \item We aim to implement the most pressing desired features by Wednesday evening/Thursday morning. This allows us to request less formal impression and usability feedback to determine whether we are taking the right direction. She has also been able to demo the application with some members of her Key Stage 3 classes so that we get an idea of how we are doing on the student side as well.
  \item Thursday and Friday are then spent refining the new features in preparation for the following week.
\end{itemize}

\subsection{Diagrammatic Evidence of User Feedback}

Figure \ref{fig:mockup} shows a boat.

\begin{figure}
  \begin{center}
  \end{center}
  \label{fig:mockup}
\end{figure}


\section{Appendix}

\subsection{Personas}

\subsection{Stakeholder Diagrams}

\subsection{System Diagrams}

\end{document}
