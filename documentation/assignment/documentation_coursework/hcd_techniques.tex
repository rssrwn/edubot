\documentclass[a4wide, 11pt]{article}
\usepackage{a4, fullpage,hyperref, graphicx}
\setlength{\parskip}{0.3cm}
\setlength{\parindent}{0cm}
\graphicspath{ {./images/} }

% This is the preamble section where you can include extra packages etc.

\begin{document}

\title{Human Centred Design Techniques Report}

\author{Group 14}

\date{\today}         % inserts today's date

\maketitle            % generates the title from the data above

% In a few short paragraphs you should discuss and reflect on the Human Centred Design techniques
% that you employed as part of the project. In particular you should discuss your initial research
% into your problem statement (identification of the current state, key design insights and your
% proposed future state) and the HCD Research Methodology employed during the project iterations.
% Additionally, you should diagrammatically evidence the impact of user feedback on your project
% direction during each of the 3 implementation weeks.

\section{HCD Techniques}

\subsection{Initial Research into Problem Statement} 

We initially decided we wanted to address a problem in the field of education and felt that, as computing students, we would be well primed to produce a product that solved a problem with computing education. Our experiences led us to believe that an appreciation for programming and computer technology is important, so we looked into the state of computing education in the UK.

In investigating this state, we found that the UK computing curriculum has undergone large changes in recent times. We dug into this further, and discovered several articles highlighting problems with the changes. Notably: information and communications (ICT) courses are being phased out in favour of computing, and ICT teachers are being forced to teach computing (a much more technical course). This puts a lot of pressure on teachers as the programming element of the course is much harder to grasp than the more lightweight elements of ICT.

Additionally, students are now taught computing from 11-14 as a compulsory course. This presents a problem, as the technical nature of the programming component may discourage students who are simply taking it because they have to (rather than out of interest).

We decided that we could potentially develop an application that makes it easier for teachers without a strong appreciation for programming to teach it, and that engages the less interested students. We envisaged a web application with two portals, one for the teacher and one for the students, whereby the teacher can set programming based exercises for their classes to complete in lesson.

Our research of existing tools for aiding a programming education helped us identify several issues with their current state:
\begin{itemize}
  \item Visual and engaging coding platforms such as Scratch do not give students the direction or programming context that the classroom requires.
  \item More directed and extensive platforms such as Code Academy do not provide facilities for teachers to see student solutions, guide them or provide feedback.
  \item Due to the technical nature of coding, there are a large number of students that may become disenchanted with it when they take is a compulsory course in their early secondary education.
\end{itemize}}

We aimed to address these issues like so:
\begin{itemize}
  \item We need to make a tool that is compelling and visual in order to capture the attention of students.
  \item However, it must take students throughout 
\end{itemize}}  

\subsection{HCD Research Methodology}

\subsection{Diagrammatic Evidence of User Feedback}

Figure \ref{fig:mockup} shows a boat.

\begin{figure}
  \begin{center}
  \includegraphics[]{placeholder.png}
  \end{center}
  \label{fig:mockup}
\end{figure}


\section{Appendix}

\subsection{Personas}

\subsection{Stakeholder Diagrams}

\subsection{System Diagrams}

\end{document}
