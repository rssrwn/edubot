\documentclass[a4wide, 11pt]{article}
\usepackage{a4, fullpage}
\setlength{\parskip}{0.3cm}
\setlength{\parindent}{0cm}

% This is the preamble section where you can include extra packages etc.

\begin{document}

\title{\vspace{-1cm}Law Coursework}

\author{\vspace{-1cm}Group 14}

\date{\today}         % inserts today's date

\maketitle            % generates the title from the data above

\renewcommand\thesection{\arabic{section}}
\renewcommand\thesubsection{\thesection.\alph{subsection}}

\section{Max Planck Institute For Meteorology Licence}

Terms 1 and 2 serve mainly to define language, and do not restrict the four freedoms.

Term 3 violates freedom 0 because it disallows use of the software for commercial purposes, in a modified form or otherwise. It could be interpreted as violating freedom 2 and 3; however, it does not explicitly mention distribution for commercial purposes, so we do not believe it violates these.

Term 4 violates freedom 2 as it says you are "not allowed to sell this “Software” or any part of it.". It restricts how you may distribute the software. For example, it may be unfeasible to distribute without charging, so this term could potentially limit your ability to help others by sharing the software. It goes on to fully violate freedoms 2 and 3 by disallowing distribution of the software as a whole, original or modified. However, it does allow sharing among colleagues and limited distribution of modular modifications.

For term 5 we assume that the provided definition of a modification applies in term 4 (where personal modification is allowed). Hence, it violates freedom 1 as it limits how you may change the program, even for personal use. The requirements to add notices to your changes do not seem to breach the principle behind the freedoms. However, the Free Software Foundation believes you should not need to notify the creators when you make changes, so it violates freedom 3. The last paragraph of this term places limitations on distribution and usage of independent new components of the models, as it restricts them in the same way as the licence restricts the existing components. This does not directly violate the four freedoms but instead the principles that govern them. It also enforces that your code be free, restricting your freedom to distribute the code as you wish.

Term 6 does not explicitly violate the freedoms, as they refer to software rather than publications. However, one of the tenets of the four freedoms is that you may make modifications without notifying the original creators, so the requirement to notify the creators upon publishing results from the software may not be in the spirit of the freedoms.

Terms 7, 8, 9, 10, 11 and 12 do not appear to violate any of the four freedoms, as they do not explicitly restrict use or regard distribution beyond the other terms.

In term 13, we take the use of "shall" in "This license shall be governed by [...] Germany" to be integrating the export control regulations of the specified region as part of the licence. In the freedoms, it is intended that, although the licences are subject to the laws of their region, those laws are not included in the licence itself.

\section{Cambridge Analytica / Facebook Story}

\subsection{Relevant Data Protection Law Requirements}

\subsection{What Could Facebook Have Done Differently?}

\subsection{Recommendations}

\end{document}
